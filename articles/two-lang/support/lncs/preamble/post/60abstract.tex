\begin{abstract}
  \textbf{Background.} By the means of the method of stochastization
  of one-step processes we get the simplified mathematical model of
  the original stochastic system. We can explore these models by
  standard methods, as opposed to the original system. The process of
  stochastization depends on the type of the system under study.
  % 
  \textbf{Purpose.} We want to get a unified abstract formalism for
  stochastization of one-step processes. This formalism should be
  equivalent to the previously introduced.
  % 
  \textbf{Methods.} To unify the methods of construction of the master
  equation, we propose to use the diagram technique.
  % 
  \textbf{Results.}  We get a diagram technique, which allows to unify
  getting master equation for the system under study.  We demonstrate
  the equivalence of the occupation number representation and the
  state vectors representation by using a Verhulst model.
  % 
  \textbf{Conclusions.} We have suggested a convenient diagram
  formalism for unified construction of stochastic systems.
  % 
  \keywords{occupation numbers representation, Fock space, Dirac
    notation, one-step processes, master equation, diagram technique}
\end{abstract}

