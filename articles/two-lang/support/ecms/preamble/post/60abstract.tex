\Section{KEYWORDS}
Hybrid modeling, fluid model, active queue management, random early detection, Modelica

\Section{ABSTRACT}

For the study and verification of our
mathematical model of RED-like active traffic management module a
discrete simulation model and a continuous analytical model were developed. However, for
various reasons, these implementations are not entirely satisfactory.
It is necessary to develop a more adequate
simulation model, possibly using a different modeling paradigm.
In order to modeling of the TCP source, the RED
control module, and the process of their interaction it is proposed to
use a hybrid (continuous-discrete) approach. For computer
implementation of the model the physical modeling language
Modelica is used. Because the language Modelica has multiple
implementations we have selected the OpenModelica compiler.
The hybrid approach allows us to take into account
the transitions between different states in the continuous model of
the TCP protocol. The hybrid approach simplified the consideration of the model due to the conversion of a
differential inclusions into a set of differential equations with
discrete transitions.
The considered approach allowed to obtain a
simple simulation model of interaction between RED module and TCP
source. This model has great potential for expansion. It is possible
to implement different types of TCP and RED. Furthermore, it is
possible to use a hybrid approach not only for the simulation but also
for analytical modeling.
