\abstract{%
\textsc{Background.} 
Presentation of probability as an intrinsic property of the nature
leads researchers to switch from deterministic to stochastic
description of the phenomena.  On the basis of the ideology of
N.\,G.~van~Kampen and C.\,W.~Gardiner the procedure of stochastization
of one-step process was formulated.  It allows to write down the
master equation based on the type of of the kinetic equations
(equations of interactions) and assumptions about the nature of the
process (which may not necessarily birth--death process).  The
kinetics of the interaction has recently attracted attention because
it often occurs in the physical, chemical, technical, biological,
environmental, economic, and sociological systems.  However, there are
no general methods for the direct study of this equation.  The
expansion of the equation in a formal Taylor series (so called
Kramers--Moyal's expansion) is used in the procedure of
stochastization of one-step processes.  It is also possible to apply
system size expansion (van Kampen's expansion).  Leaving in the
expansion terms up to  the second order we can get the Fokker--Planck
equation, and thus the Langevin equation.  It should be clearly
understood that these equations are approximate recording of the
master equation.
\textsc{Purpose.} 
However, this does not eliminate the need for the study of the master
equation. Moreover, the power series produced during the master equation
decomposition may be divergent (for example, in
spatial models). This makes it impossible to apply the classical
perturbation theory.
\textsc{Method.}
It is proposed to use quantum field perturbation theory for the
statistical systems (so-called Doi method).  The perturbation series
are treated in the spirit of the Feynman path integral, where the Green's
functions of the perturbed Liouville operator of the master equation
are propagators.  For more convenience of selection of the perturbed
and unperturbed parts of the Liouville operator and to obtain the
explicit form of the Green function of the master equation we need to
rewrite the equation in the occupation number representation (Fock
state).
\textsc{Results.}
This work is a methodological material that describes the principles
of master equation solution based on quantum field
perturbation theory methods.  The characteristic property of the this work is that it is intelligible
for non-specialists in quantum field theory.  As an example the Verhulst model is used
because of its simplicity and clarity (the first order
equation, is independent of the spatial variables, however, contains
non-linearity).
\textsc{Conclusions.}
We show a full equivalence of operator and combinatorial methods of obtaining and study 
of the one-step process master equation.
}
%
\maketitle
