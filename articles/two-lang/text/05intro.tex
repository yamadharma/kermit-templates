\begin{Russian}
\section{Введение}
\end{Russian}
\begin{English}
\section{Introduction}
\end{English}
\label{sec:intro}

\begin{Russian}
При моделировании технических систем с управлением возникает
необходимость исследования их характеристик.
Также необходимо исследование влияния параметров систем на эти
характеристики.
В системах с управлением возникает такое паразитное явление, как
автоколебательный режим.
Нами проводились исследования по определению области
возникновения автоколебаний. Однако параметры этих автоколебаний нами
не исследовались. В данной статье мы предлагаем использовать метод
гармонической линеаризации для данной задачи. Этот метод применяется в
теории управления, однако данный раздел математики достаточно редко
используется в классическом математическом моделировании. Авторы
предлагаю методическую статью, призванную познакомить неспециалистов с
применением этого метода.
\end{Russian}
\begin{English}
While modeling technical systems with control it is often required to 
study characteristics of these systems. Also it is necessary to study the influence 
of system parameters on characteristics. In systems with control 
there is a parasitic phenomenon as self-oscillating mode.
We carried out 
studies to determine the region of the self-oscillations emergence. 
However, the parameters of these oscillations were not investigated. 
In this paper, we propose to use the harmonic 
linearization method for this task. This method is used in control 
theory, but this branch of mathematics rarely used in classical 
mathematical modeling. The authors offer a methodological article in order  
to introduce this method to non-specialists. 
\end{English}

\begin{Russian}
\end{Russian}
\begin{English}
\end{English}

% \begin{Russian}
% В~разделе~\ref{sec:notation} даются основные обозначения и соглашения,
% применяемые в статье. В разделе~\ref{sec:maxwell_curv} вводятся основные
% соотношения для уравнений Максвелла в криволинейных координатах (для
% более подробного ознакомления можно обратиться к другим статьям
% авторов~\cite{kulyabov:2011:vestnik:curve-maxwell,
%   kulyabov:2012:vestnik:2012-1}). В разделе~\ref{sec:formal_geometr}
% приводятся собственно
% расчёты по геометризации Плебаньского.
% \end{Russian}
% \begin{English}
% In paragraph~\ref{sec:notation} we prosecuted provides basic notation and conventions
% used in the article. In paragraph~\ref{sec:maxwell_curv} are the main
% relations for the Maxwell's equations in curvilinear coordinates (for
% more detailed discussion the reader can be refer to other 
% authors articles~\cite{kulyabov:2012:vestnik:2012-1}). In paragraph~\ref{sec:formal_geometr} are presented actual
% calculations on Plebanski geometrization.
% \end{English}

% \begin{Russian}
% \section{Обозначения и соглашения}
% \end{Russian}
% \begin{English}
% \section{Notations and conventions}
% \end{English}
% \label{sec:notation}


% \begin{Russian}
%   \begin{enumerate}

%   \item Будем использовать нотацию абстрактных
%     индексов~\cite{penrose-rindler:spinors::ru}. В данной нотации тензор как
%     целостный объект обозначается просто индексом (например, $x^{i}$),
%     компоненты обозначаются подчёркнутым индексом (например,
%     $x^{\crd{i}}$).

%   \item Будем придерживаться следующих соглашений.  Греческие индексы
%     ($\alpha$, $\beta$) будут относиться к четырёхмерному
%     пространству и в компонентном виде будут иметь следующие значения:
%     $\crd{\alpha} = \overline{0,3}$. Латинские индексы из середины
%     алфавита ($i$, $j$, $k$) будут относиться к трёхмерному
%     пространству и в компонентном виде будут иметь следующие значения:
%     $\crd{i} = \overline{1,3}$.

%   \item Запятой в индексе обозначается частная производная по
%     соответствующей координате ($f_{,i} := \partial_{i} f$);
%     точкой с запятой --- ковариантная производная ($f_{;i} := \nabla_{i}
%     f$).

%   \item Для записи уравнений электродинамики в работе используется
%     система СГС симметричная~\cite{sivukhin:1979:ufn::ru}.

%   \end{enumerate}
% \end{Russian}
% \begin{English}
%   \begin{enumerate}

%   \item We will use the notation of abstract
%     indices~\cite{penrose-rindler:spinors::en}. In this notation tensor
%     as a
%     complete object is denoted merely by an index (e.g., $x^{i}$). Its
%     components are
%     designated by underlined indices (e.g., $x^{\crd{i}}$).
    
%   \item We will adhere to the following agreements. Greek indices
%     ($\alpha$, $\beta$) will refer to the four-dimensional space, in the
%     component form it looks like: $\crd{\alpha} = \overline{0,3}$.  Latin
%     indices from the middle of the alphabet ($i$, $j$, $k$) will refer
%     to the three-dimensional space, in the component form it looks like:
%     $\crd {i} = \overline{1,3}$.
    
%   \item The comma in the index denotes a partial derivative with respect to
%     corresponding coordinate ($f_{, i} := \partial_{i}f$); semicolon
%     denotes a covariant derivative ($f_{;i} := \nabla_{i} f$).

%   \item The CGS symmetrical system~\cite{sivukhin:1979:ufn::en} is used for 
%   notation of the equations of electrodynamics.

%   \end{enumerate}
% \end{English}

%%% Local Variables:
%%% mode: latex
%%% coding: utf-8-unix
%%% End:
