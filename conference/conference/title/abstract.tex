\thispagestyle{empty}
{%
\selectlanguage{russian}%
\small%

\begin{flushleft}
УДК 004.7:004.4.001:621.391:007(063)\\
ББК 32.973.202:32.968\\
\quad\quad P~24
\end{flushleft}

\bigskip

\noindent
\begin{minipage}[t]{0.1\linewidth}
P~24
\end{minipage}
\hfill
\begin{minipage}[t]{0.85\linewidth}
\textbf{Распределенные компьютерные и телекоммуникационные
сети: управление, вычисление, связь (DCCN-2016)}~=
\textbf{Distributed computer and communication networks: control, computation, communications
(DCCN-2016)}~:
материалы Девятнадцатой международной научной
конференции, 21--25 нояб. 2016~г. : в 3~т.;
под общ.~ред. В.\,М.~Вишневского и К.\,Е.~Самуйлова~--- М.:~РУДН, 2016.

\noindent
ISBN \isbn

\noindent
T.~\volumeNum : \volume~= \volumeEn.~--- 
\numpages{}~с.~:~ил.


\noindent
ISBN \isbnVol
\end{minipage}

\noindent
\grant

\bigskip

В научном издании представлены материалы
Девятнадцатой международной научной
конференции «Распределенные компьютерные и телекоммуникационные сети:
управление, вычисление, связь» по следующим направлениям:
\begin{itemize}
\item Оптимизация архитектуры компьютерных и телекоммуникационных сетей;
\item Управление в компьютерных и телекоммуникационных сетях;
\item Оценка производительности и качества обслуживания в беспроводных сетях;
\item Аналитическое и имитационное моделирование коммуникационных систем последующих поколений;
\item Беспроводные сети 4G/5G и технологии сантиметрового и миллиметрового диапазона радиоволн;
\item RFID-технологии и их применение в интеллектуальных транспортных сетях;
\item Интернет вещей, носимые устройства, приложения распределенных информационных систем;
\item Распределенные системы и системы облачного вычисления, анализ больших данных;
\item Вероятностные и статистические модели в информационных системах;
\item Теория очередей, теория надежности и их приложения;
\item Математическое моделирование высокотехнологичных систем;
\item Математическое моделирование и задачи управления.
\end{itemize}

Сборник материалов конференции предназначен для научных работников и специалистов в области теории и
практики построения компьютерных и телекоммуникационных сетей. 

\bigskip

\begin{center}
Текст воспроизводится в том виде, в котором представлен авторами
\end{center}


\bigskip

\begin{center}
\textbf{Утверждено к печати Программным комитетом конференции}
\end{center}

\bigskip

\noindent
\begin{minipage}[t]{0.35\linewidth}
\noindent
ISBN \isbnVol\\
ISBN \isbn
\end{minipage}
\hfill
\begin{minipage}[t]{0.6\linewidth}
\noindent
\copyright Коллектив авторов, \year\\
\copyright Российский университет дружбы народов, \year
\end{minipage}

} %selectlanguage

\clearpage

\setcounter{page}{3}
